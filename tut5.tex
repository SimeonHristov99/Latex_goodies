\documentclass[11pt]{article}

\title{My \LaTeX\ Document}
\author{John Bobson}
\date{\today}
%\date{July 26, 202}

\begin{document}
\tableofcontentsk
\maketitle

This will produce \textit{italicized} text.

This will produce \textbf{bold face} text.

This will produce \textsc{small caps} text.

This will produce \texttt{typewriter font} text.

\vspace{1cm}

Please excuse my dear aunt Sally.

Please excuse my \begin{large}dear aunt Sally\end{large}.

Please excuse my \begin{Large}dear aunt Sally\end{Large}.

Please excuse my \begin{huge}dear aunt Sally\end{huge}.

Please excuse my \begin{Huge}dear aunt Sally\end{Huge}.

Please excuse my dear aunt Sally. This is the same as: Please excuse my \begin{normalsize}dear aunt Sally\end{normalsize}.

Please excuse my \begin{small}dear aunt Sally\end{small}.

Please excuse my \begin{scriptsize}dear aunt Sally\end{scriptsize}.

Please excuse my \begin{tiny}dear aunt Sally\end{tiny}.

\vspace{1cm}

\begin{center}
This line is centered.
\end{center}

\begin{flushleft}
This line is left-justified.
\end{flushleft}

\begin{flushright}
This line is right-justified.
\end{flushright}

% Or alternatively:

\centering
This line is centered.

This line is left-justified.

This line is right-justified.

\Large
This line is centered.

This line is left-justified.

This line is right-justified.

\tiny
This line is centered.

This line is left-justified.

This line is right-justified.

\section{Linear Functions}
	\subsection{Slope-Intercept Form}
		\subsubsection{Example 1:}
		\subsubsection{Example 2:}	
	\subsection{Standard Form}
	\subsection{Point-Slope Form}
\section{Quadratic Functions}
	\subsection{Vertex Form}
	\subsection{Standard Form}
	\subsection{Factored Form}

\end{document}
