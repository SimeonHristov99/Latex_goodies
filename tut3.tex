\documentclass[11pt]{article}
\usepackage{amsfonts, amssymb, amsmath}
\usepackage{float}
\parindent 0px % turn off indenting
\pagestyle{empty}

\begin{document}

% \\ is a soft return. It does not create a new paragraph.
% if you leave 1 empty line, it's a new paragraph.

The distributive property states that $a(b+c)=ab+ac$, for all $a, b, c \in \mathbb{R}$.\\[6pt]
The equivalence class of $a$ is $[a]$.\\[6pt]
The set $A$ is defined to be $\{1, 2, 3\}$.\\[6pt]
The movie ticket costs $\$11.50$.

$$2\left(\frac{1}{x^2 - 1}\right)$$
$$2\left[\frac{1}{x^2 - 1}\right]$$
$$2\left\{\frac{1}{x^2 - 1}\right\}$$
$$2\left \langle   \frac{1}{x^2 - 1}   \right \rangle$$
$$2\left | \frac{1}{x^2 - 1} \right |$$

$$\left.\frac{dy}{dx}\right|_{x=1}$$

$$\left( \frac{1}{1+\left(\frac{1}{1 + x}\right)} \right)$$

Tables: \\

% c stands for centered content
% l stands for left alligned content
% the number of letters is the number of columns
% | stands for lines in between columns
% hline is for horizontal lines

\begin{tabular}{|c|c|c|c|c|c|}
\hline
$x$ & 1 & 2 & 3 & 4 & 5 \\ \hline
$f(x)$ & 10 & 11 & 12 & 13 & 14 \\ \hline
\end{tabular}

\vspace{1cm}

\begin{table}[H]
\centering
\def\arraystretch{1.5}
\begin{tabular}{|c||c|c|c|c|c|}

\hline
$x$ & 1 & 2 & 3 & 4 & 5 \\ \hline
$f(x)$ & $\frac{1}{2}$ & 11 & 12 & 13 & 14 \\ \hline

\end{tabular}
\caption{These values represent the function $f(x)$.}
\end{table}


\begin{table}[H]
\centering
\caption{The relationship between $f$ and $f'$.}
\def\arraystretch{1.5}
\begin{tabular}{|l|p{3in}|}

\hline
$f(x)$ & $f'(x)$ \\ \hline
$x > 0$ & The function $f(x)$ is increasing. The function $f(x)$ is increasing. The function $f(x)$ is increasing. The function $f(x)$ is increasing. \\ \hline

\end{tabular}
\end{table}

Arrays:
\begin{align}
% \, enforces space (as they get ignored in math mode)
% 5x^2\, \text{place your words here}
% 5x^2\, \text{place your words here}
5x^2 - 9 = x + 3\\
5x^2 - x - 12 = 0
\end{align}

\begin{align*}
5x^2 - 9 &= x + 3\\
5x^2 - x - 12 &= 0\\
&=12 + x - 5x^3
\end{align*}

\begin{align}
% \, enforces space (as they get ignored in math mode)
% 5x^2\, \text{place your words here}
% 5x^2\, \text{place your words here}
5x^2 - 9 = x + 3\\
5x^2 - x - 12 = 0
\end{align}

\end{document}
